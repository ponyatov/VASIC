\input{../texheader/mini}
\title{VASIC}
\author{\copyright\ Dmitry Ponyatov \email{dponyatov@gmail.com}}
\newcommand{\vasic}{$VASIC$}
\begin{document}
\maketitle
\tableofcontents\secdown

\secly{Введение}

\vasic\ --- диалект языка BASIC, реализация основана на \href{https://www.youtube.com/playlist?list=PLBOh8f9FoHHiKx3ZCPxOZWUtZswrj2zI0}{видеокурсе по созданию
интерпретатора BASIC}\ и кое-каких собственных наработах по 
\term{лексическому программированию}.
Синтаксис построен на основе видеокурса, но сделаны некоторые заимствования из
\py\ и функциональных языков.

\emph{Ядро написано на стандартном \ci, специально для кросс-компиляции и выполнения
интерпретатора на микроконтроллерах}. \cpp\ очень удобен для компьютеров, 
но для встраиваемых
применений требует крайне профессиональной и аккуратной
адаптации \cpp-кода и библиотек поддержки\note{\prog{newlib}, менеджер 
динамической памяти, \cpp-runtime}\ для выполнения на микроконтроллерах.
Эта проблема связана с критически маленьким объемом ОЗУ, и очень активным 
использованием динамической памяти программами на \cpp.

\secly{\file{lex skeleton}: скелет лексической программы (\ci-вариант)}

\begin{tabular}{l l l}
	src.src & & исходник тестового скрипта, проверяет все функции интерпретатора\\
	log.log & & лог выполнения теста в интерпретаторе \\
	\hline
	y.y & \prog{yacc/bison} & синтаксичский анализатор: грамматика языка \\
	l.l & \prog{lex/flex} & лексер: распознавание единичных элементов \term{токенов}\\
	h.h & \cpp & хедеры общие для всех модулей кода (лексер/парсер/ядро) \\
	c.c & \cpp & ядро языка, реализация всех объектов языка на ANSI \ci\\
	Makefile\ \ref{make} & \prog{make} & скрипты сборки проекта утилитой \prog{make}\\
	\hline
bat.bat & \vim & запускалка редактора \vim\ под \win\\
.gitignore & \prog{git} & маски файлов, не включаемых в
\prog{git}-репозиторий \\
README.md & github & описание проекта \\
manual.tex & \LaTeX & исходный текст этого руководства на языке верстки \LaTeX \\
\end{tabular}

\prog{git}\note{система управления версиями исходного текста, 
и сетевое хранилище \href{https://github.com/ponyatov/VASIC}{GitHub}}

\secrel{Синтаксис языка \vasic}\secdown
\secrel{Комментарии}

\begin{verbatim}
# строчный комментарий, от # до конца строки
\end{verbatim}

\secup

\secrel{Исходный код ядра}\secdown
\secrel{Makefile}\label{make}
\lstv{Makefile}{Makefile}

\secup

\end{document}
