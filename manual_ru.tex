\input{../texheader/mini}
\newcommand{\vasic}{\textbf{VASIKЪ}}
\title{\vasic}
\author{\copyright\ Dmitry Ponyatov \email{dponyatov@gmail.com}, GNU Lesser GPL\\
институт комического приборостроения СГОУ}
\begin{document}
\maketitle

\input{../texheader/lgpl}

\tableofcontents\secdown

\secly{Введение}

\vasic\ --- диалект языка $BASIC$, реализация основана на \href{https://www.youtube.com/playlist?list=PLBOh8f9FoHHiKx3ZCPxOZWUtZswrj2zI0}{видеокурсе по созданию
интерпретатора BASIC}\ и кое-каких собственных наработках по 
\term{лексическому программированию}.
Синтаксис построен на основе видеокурса, но сделаны некоторые заимствования из
\py\ и функциональных языков.

\emph{Ядро написано на стандартном ANSI \ci\note{gcc -ansi\ , специально чтобы 
можно было компилировать код другими кросс-компиляторами: для микроконтроллеров,
не входящих в список gcc targets, и коммерческими типа IAR, Keil,
Intel, OpenWatcom, Visual C,..},
специально для кросс-компиляции и выполнения
интерпретатора на микроконтроллерах}. \cpp\ очень удобен для компьютеров, 
но для встраиваемых
применений требует крайне профессиональной и аккуратной
адаптации \cpp-кода и библиотек поддержки\note{\prog{newlib}, менеджер 
динамической памяти, \cpp-runtime}\ для выполнения на микроконтроллерах.
Эта проблема связана с критически маленьким объемом ОЗУ, и очень активным 
использованием динамической памяти программами на \cpp.

\secly{\file{lex skeleton}: скелет лексической программы (\ci-вариант)}

\begin{tabular}{l l l}
	\file{test/}\ $\rightarrow$\ \file{src.src} && 
	исходники отдельных тестов интерпретатора \\
	log.log & & лог выполнения теста в интерпретаторе \\
	\hline
	y.y & \prog{yacc/bison} & синтаксичский анализатор: грамматика языка \\
	l.l & \prog{lex/flex} & лексер: распознавание единичных элементов \term{токенов}\\
	h.h & \cpp & хедеры общие для всех модулей кода (лексер/парсер/ядро) \\
	c.c & \cpp & ядро языка, реализация всех объектов языка на ANSI \ci\\
	Makefile\ \ref{make} & \prog{make} & скрипты сборки проекта утилитой \prog{make}\\
	\hline
bat.bat & \vim & запускалка редактора \vim\ под \win\\
.gitignore & \prog{git} & маски файлов, не включаемых в
\prog{git}-репозиторий \\
README.md & github & описание проекта \\
manual.tex & \LaTeX & исходный текст этого руководства на языке верстки \LaTeX \\
\end{tabular}

\prog{git}\note{система управления версиями исходного текста, 
и сетевое хранилище \href{https://github.com/ponyatov/VASIC}{GitHub}}

\secly{Загрузка и сборка}

Пакет предоставляется \emph{в исходных текстах}\ с репозитория \url{https://github.com/ponyatov/VASIC}. 
\input{../texheader/soft}

\secrel{Синтаксис языка \vasic\\(с примерами кода из батареи тестов)}\secdown
\secrel{Комментарии}

\lstv{test/comment.src}{test/comment.src}
\lstv{}{tmp/comment.log}

\secrel{Скаляры}\secdown

\termdef{Скаляры}{скаляр}\ --- типы объектов, соответствующие \emph{единичным}\
самодостаточным элементам данных, которые можно считать атомами
информатики\note{computer science}: символ (идентификатор), строка, число.

\secrel{\class{sym}: символ (идентификатор)}\label{sym}

\lstv{test/sym.src}{test/sym.src}
\lstv{}{tmp/sym.log}

\secrel{\class{str}: строка}\label{str}

\lstv{test/str.src}{test/str.src}
\lstv{}{tmp/str.log}

\secrel{\class{int}: целое число}\label{int}

\lstv{test/int.src}{test/int.src}
\lstv{}{tmp/int.log}

\secrel{\class{num}: число с плавающей точкой}\label{num}

\lstv{test/num.src}{test/num.src}
\lstv{}{tmp/num.log}

\secrel{\class{hex}: машинное целое}\label{hex}

\lstv{test/hex.src}{test/hex.src}
\lstv{}{tmp/hex.log}

\secrel{\class{bin}: битовая строка}\label{bin}

\lstv{test/bin.src}{test/bin.src}
\lstv{}{tmp/bin.log}

\secup

\secrel{Команды}\secdown

\secrel{\var{PRINT}}\label{print}

Реализация команды \var{print}\ рассмотрено в уроке
\href{https://www.youtube.com/watch?v=LDDRn2f9fUk&list=PLBOh8f9FoHHiKx3ZCPxOZWUtZswrj2zI0&index=2}{Make Your Own Programming Language - Part 1 - Lexer}

\lstv{test/print.src}{test/print.src}
\lstv{}{tmp/print.log}

\secup

\secup

\secrel{Исходный код ядра}\secdown
\secrel{Makefile}\label{make}
\lstv{Makefile}{Makefile}

\secup

\end{document}
